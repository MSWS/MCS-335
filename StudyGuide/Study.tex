\documentclass[10pt]{article}

\usepackage[tmargin=0.2in, bmargin=0.25in, lmargin = 0.1in, rmargin = 0.25in]{geometry}
\usepackage{amsmath, amssymb, amsthm}
\usepackage{enumitem}
\usepackage{multicol}
\usepackage{listings}
\usepackage{pstricks}
\usepackage{calc}

\begin{document}


\begin{multicols*}{2}
    \begin{minipage}{\columnwidth}
        \subsection*{Optimal Rays}
        \begin{tabular}{c|ccccc}
            -9 & 0 & 1 & 0 & 0  & 0 \\
            \hline
            2  & 1 & 1 & 1 & -1 & 0 \\
            6  & 0 & 1 & 2 & -1 & 1
        \end{tabular}
        Note that our BFS \(\overline{x}_0^* = [2, 0, 0, 0, 6]\)
        Since \(x_4\) has \(c_k = 0, \forall a_{ik} \leq 0\), we know there is an optimal ray.
        \begin{enumerate}
            \item \(x_2, x_3\) are non-basic and not our optimal ray column, so set those to 0 and solve for the basic variable.
                  \begin{align*}
                      2 = x_1 + x_2 -x_4 \rightarrow x_1 = 2 + x_4 \\
                      6 = x_2 + 2x_3 - x_4 \rightarrow x_5 = 6 + x_4
                  \end{align*}
            \item Set \(x_4 = t, t \geq 0\)
            \item We get
                  \begin{align*}
                      \overline{x}(t) & = [2 + t, 0, 0, t, 6 + t], t \geq 0             \\
                                      & = [2, 0, 0, 0, 6] + t [1, 0, 0, 1, 1], t \geq 0
                  \end{align*}
                  which is an \textbf{optimal ray}.
        \end{enumerate}
        Pivoting in \(x_3\):

        \begin{tabular}{c|ccccc}
            -9 & 0 & 1 & 0 & 0  & 0 \\
            \hline
            2  & 1 & 1 & 1 & -1 & 0 \\
            6  & 0 & 1 & 2 & 1  & 1
        \end{tabular}
        We \textit{can} still pivot on \(x_4\), but first our BFS \(\overline{x}_1^* = [0, 0, 2, 0, 2]\).

        Now, pivoting gets us

        \begin{tabular}{c|ccccc}
            -9 & 0  & 1  & 0 & 0 & 0 \\
            \hline
            4  & -1 & 0  & 1 & 0 & 1 \\
            2  & -2 & -1 & 0 & 1 & 1
        \end{tabular}

        This BFS \(\overline{x}_2^* = [0, 0, 4, 2, 0]\), which is \(\neq \overline{x}_1^*\).
        We also have an \textit{optimal ray} here, so following the process above...

        Set \(x_2 = x_5 = 0\), solve for \(x_3\) and \(x_4\) respectively.
        \begin{align*}
            x_3 & = 4 + x_1  \\
            x_4 & = 2 + 2x_1
        \end{align*}
        Set \(x_1 = t, t \geq 0\)
        \begin{align*}
            \overline{x}(t) & = [t, 0, 4 + t, 2 + 2t, 0], t \geq 0            \\
                            & = [0, 0, 4, 2, 0] + t [1, 0, 1, 2, 0], t \geq 0
        \end{align*}
        So we've found all possible BFS and optimal rays.
        Thus, our \textbf{optimal set} is
        \textit{all possible convex combinations of pair of points from edges}
        \([\overline{x}_0^*, \overline{x}_1^*]\) and \([\overline{x}_1^*, \overline{x}_2^*]\)
        and optimal rays \(\overline{x}_0^* + t[1, 0, 0, 1, 1]\) and \(\overline{x}_2^* + t[1, 0, 1, 2, 0]\) with \(t \geq 0\).
        \subsection*{Graphing}
        \begin{minipage}{0.4\columnwidth}
            \begin{tabular}{c|cccccc}
                * & * & * & *  & * & * & *  \\
                \hline
                2 & 0 & 2 & 1  & 1 & 0 & -1 \\
                8 & 0 & 4 & -2 & 0 & 1 & 1  \\
                6 & 1 & 1 & 2  & 0 & 0 & 1
            \end{tabular}
        \end{minipage}
        \columnbreak
        \begin{minipage}{0.7\columnwidth}
            \begin{enumerate}
                \item Pick non-basic var to use as vert axis.
                \item Solve equations for basic variable.
                \item Set that equation \(= 0\).
            \end{enumerate}
        \end{minipage}
        \begin{align*}
            2   & = 2x_2 + x_3 + x_4 - x_6                \\
            x_4 & = 2 - 2x_2 - x_3 + x_6 \mathbf{= 0}     \\
            x_6 & = 2x_2 + x_3 + x_4 - 2                  \\
            % & \vdots                                  \\
                & \mathrel{\makebox[\widthof{=}]{\vdots}} \\
            x_6 & = 8 - 4x_2 + 2x_3                       \\
            x_6 & = 6 - x_2 - 2x_3
        \end{align*}
        Good Luck! - Isaac Boaz \today
    \end{minipage}
    \columnbreak \\
    \begin{minipage}{\columnwidth}
        \subsection*{Subproblem Technique}
        \begin{enumerate}
            \item Pivot to get \(I_m\) with 0s in the OF row.
                  \begin{itemize}
                      \item If you get a row of 0s, delete and continue
                      \item If you get a row with \(b_i \neq 0\) and \(a_{ik} = 0\) for all \(k\) - Infeasible Form 1
                  \end{itemize}
                  Once you have \(I_m\) with 0s in the OF row, you're done with Step 0.
                  \\
                  Either you're in CF or not.
                  \begin{itemize}
                      \item If CF, done with step 1.
                      \item If not, some \(b_i < 0\).
                  \end{itemize}
            \item Try to get all \(b_i \geq 0\) \\
                  If \textit{all} \(b_i < 0\), pick \textit{any negative} entry in any row and pivot. You will now have some \(b_i > 0\).
                  \begin{itemize}
                      \item If some \(b_i < 0\) and all \(a_{ik} \geq 0\) - Infeasible Form 2
                  \end{itemize}
            \item Now you should have at least one row with \(b_i \geq 0\) and
                  at least one row with \(b_i < 0\). Form the subproblem and pivot
                  to either optimal/unbounded form or until \(b_i\) becomes non-negative.
                  Make sure you zero-out the entire column when you pivot, so at the end you
                  still have \(I_m\) with 0s in OF row.
                  \begin{itemize}
                      \item If optimal form for subproblem and \(b_i < 0\) - Infeasible Form 2
                      \item If optimal form for subproblem and \(b_i \geq 0\), you have one fewer \(b_i < 0\). Form a new subproblem for any \(b_i < 0\), otherwise you're now in CF.
                      \item If unbounded form for subproblem and \(b_i < 0\), then pivot on a \textit{negative} entry of the subproblem OF in an unbounded column.
                  \end{itemize}
        \end{enumerate}
        \subsection*{Artificial Variables}
        \begin{enumerate}
            \item Get \(\overline{b} \geq \overline{0}\) \\
                  Multiply any rows with \(b_i < 0\) by -1.
            \item Get \(I_m\)
                  \begin{itemize}
                      \item Add \textbf{artificial vars} \(y_i\) as need to get \(I_m\).
                      \item Zero out the artificial OF row.
                      \item Perform simplex algorithm on artificial problem.
                      \item If optimal artificial OF value \(> 0\), then infeasible.
                      \item If optimal artificial OF value \(< 0\), you did something wrong.
                      \item If optimal artificial OF value \(= 0\), there are two cases:
                            \begin{enumerate}[label=(\roman*)]
                                \item All artificial vars are non-basic. \\
                                      Delete artificial OF row and replace original OF row. \\
                                      Delete columns corresponding to artificial vars.
                                      \textbf{Zero out the original OF row.}
                                \item At least one artificial var is basic. \\
                                      Pivot in any \(x_j\) column in the row corresponding to the var with non-zero entry. (If no such column exists, delete the row). \\
                                      Now you have one fewer basic artificial var. Delete the corresponding artificial column and repeat until you are in case (i).
                            \end{enumerate}
                  \end{itemize}
        \end{enumerate}
    \end{minipage}
\end{multicols*}

\end{document}